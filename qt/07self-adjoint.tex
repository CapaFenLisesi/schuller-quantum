

While we have already given some of the following definitions in the introductory section on the axioms of quantum mechanics, we reproduce them here for completeness. 

\subsection{Adjoint operators}

\bd
A linear map or operator $A\cl\mathcal{D}_A\to\mathcal{H}$ is said to be \emph{densely defined} if $\mathcal{D}_A$ is dense in $\mathcal{H}$, i.e.
\bse
\forall \, \varepsilon > 0 : \forall \, \psi\in \mathcal{H} : \exists \, \alpha \in \mathcal{D}_A : \ \|\alpha - \psi\|< \varepsilon.
\ese
\ed
Equivalently, $\overline{\mathcal{D}_A}=\mathcal{H}$, i.e.\ for every $\psi\in\mathcal{H}$ there exists a sequence $\{\alpha_n\}_{n\in\N}$ in $\mathcal{D}_A$ whose limit is $\psi$.
\bd
Let $A\cl \mathcal{D}_A\to \mathcal{H}$ be a densely defined operator on $\mathcal{H}$. The \emph{adjoint}\index{adjoint} of $A$ is the operator $A^*\cl\mathcal{D}_{A^*}\to\mathcal{H}$ defined by
\ben[label=(\roman*)]
\item $\mathcal{D}_{A^*}:=\{\psi\in\mathcal{H} \mid \exists \, \eta\in \mathcal{H} : \forall \, \alpha \in \mathcal{D}_A : \, \langle \psi | A\alpha \rangle = \langle \eta | \alpha \rangle\}$
\item $A^*\psi:=\eta$.
\een
\ed

\bp
The adjoint operator $A^*\cl\mathcal{D}_{A^*}\to\mathcal{H}$ is well-defined.
\ep

\bq
Let $\psi\in \mathcal{H}$ and let $\eta,\widetilde\eta \in \mathcal{H}$ be such that
\bse
\forall \, \alpha\in \mathcal{D}_A : \quad \langle \psi | A\alpha \rangle = \langle \eta | \alpha \rangle \ \text{ and } \ \langle \psi | A\alpha \rangle = \langle \widetilde \eta | \alpha \rangle.
\ese
Then, for all $\alpha$ in $\mathcal{D}_A$, we have
\bse
\langle \eta - \widetilde \eta | \alpha \rangle
 =  \langle  \eta | \alpha \rangle - \langle \widetilde \eta | \alpha \rangle  =    \langle \psi | A\alpha \rangle- \langle \psi | A\alpha \rangle  =  0
\ese
and hence, by positive-definiteness, $\eta = \widetilde \eta$.
\eq
If $A$ and $B$ are densely defined and $\mathcal{D}_A=\mathcal{D}_B$, then the pointwise sum $A+B$ is clearly densely defined and hence, $(A+B)^*$ exists. However, we do \emph{not} have $(A+B)^*=A^*+B^*$ in general, unless one of $A$ and $B$ is bounded, but we do have the following result.
\bp
If $A$ is densely defined, then 
\bse
(A+z\id_{\mathcal{D}_A})^*=A^*+\overline{z}\id_{\mathcal{D}_A}
\ese
for any $z\in\C$.
\ep
The identity operator $\id_{\mathcal{D}_A}$ is usually suppressed in the notation, so that the above equation reads $(A+z)^*=A^*+\overline{z}$.

\bd
Let $A\cl \mathcal{D}_A\to \mathcal{H}$ be a linear operator. The \emph{kernel}\index{kernel} and \emph{range}\index{range} of $A$ are
\bse
\ker (A)  :=  \{\alpha \in \mathcal{D}_A \mid A\alpha = 0\},\quad\qquad\ran (A)  :=  \{A\alpha \mid \alpha \in \mathcal{D}_A \}.
\ese
The range is also called \emph{image}\index{image} and $\im (A)$ and $A(\mathcal{D}_A)$ are alternative notations.
\ed

\bp
An operator $A\cl \mathcal{D}_A\to \mathcal{H}$ is
\ben[label=(\roman*)]
\item injective if, and only if, $\ker(A)=\{0\}$
\item surjective if, and only if, $\ran(A)=\mathcal{H}$.
\een
\ep

\bd
An operator $A\cl \mathcal{D}_A\to \mathcal{H}$ is \emph{invertible}\index{invertible operator} if there exists an operator $B\cl\mathcal{H}\to \mathcal{D}_A$ such that $A\circ B = \id_{\mathcal{H}}$ and $B\circ A = \id_{\mathcal{D}_A}$.
\ed

\bp
An operator $A$ is invertible if, and only if, 
\bse
\ker(A)=\{0\}\quad \text{ and }\quad\, \overline{\ran(A)}=\mathcal{H}.
\ese
\ep

\bp
\label{prp:kerran}
Let $A$ be densely defined. Then, $\ker(A^*)=\ran(A)^{\perp}$.
\ep

\bq
We have
\bse
\psi\in\ker(A^*)\ \ \Leftrightarrow\ \ A^*\psi =0\ \ \Leftrightarrow\ \ \forall \, \alpha \in \mathcal{D}_A : \langle \psi | A\alpha \rangle = 0 \ \ \Leftrightarrow\ \ \psi \in \ran(A)^{\perp} .\qedhere
\ese
\eq

\bd
Let $A\cl \mathcal{D}_A \to \mathcal{H}$ and $B\cl \mathcal{D}_B \to \mathcal{H}$ be operators. We say that $B$ is an \emph{extension}\index{extension} of $A$, and we write $A\subseteq B$, if
\ben[label=(\roman*)]
\item $\mathcal{D}_{A}\subseteq \mathcal{D}_{B}$
\item $\forall \, \alpha \in \mathcal{D}_{A} : \ A\alpha = B \alpha$.
\een
\ed

\bp
\label{prp:adjointreverse}
Let $A,B$ be densely defined. If $A\subseteq B$, then $B^*\subseteq A^*$.
\ep

\bq
\ben[label=(\roman*)]
\item Let $\psi \in \mathcal{D}_{B^*}$. Then, there exists $\eta\in\mathcal{H}$ such that
\bse
\forall \, \beta \in \mathcal{D}_B : \ \langle \psi | B\beta \rangle = \langle \eta | \beta \rangle.
\ese
In particular, as $A\subseteq B$, we have $\mathcal{D}_A\subseteq\mathcal{D}_B$ and thus
\bse
\forall \, \alpha \in \mathcal{D}_A\subseteq\mathcal{D}_B : \ \langle \psi | B\alpha \rangle = \langle \psi | A\alpha \rangle =\langle \eta | \alpha \rangle.
\ese
Therefore, $\psi \in \mathcal{D}_{A^*}$ and hence, $\mathcal{D}_{B^*}\subseteq \mathcal{D}_{A^*}$.
\item From the above, we also have $B^*\psi := \eta =: A^*\psi$ for all $\psi \in \mathcal{D}_{B^*}$.\qedhere
\een
\eq


\subsection{The adjoint of a symmetric operator}


\bd
A densely defined operator $A\cl\mathcal{D}_A\to\mathcal{H}$ is called \emph{symmetric}\index{symmetric operator} if
\bse
\forall \, \alpha,\beta \in \mathcal{D}_A : \ \langle\alpha|A\beta\rangle = \langle A\alpha|\beta\rangle.
\ese
\ed

\br
In the physics literature, symmetric operators are usually referred to as \emph{Hermitian}\index{Hermitian operator} operators. However, this notion is then confused with the that of self-adjointness when physicists say that observables in quantum mechanics correspond to Hermitian operators, which is not the case. On the other hand, if one decides to use Hermitian as a synonym of self-adjoint, it is then not true that all symmetric operators are Hermitian. In order to prevent confusion, we will avoid the term Hermitian altogether. 
\er

\bp
\label{prp:symext}
If $A$ is symmetric, then $A\subseteq A^*$.
\ep

\bq
Let $\psi\in\mathcal{D}_A$ and let $\eta:=A\psi$. Then, by symmetry, we have
\bse
\forall \, \alpha\in \mathcal{D}_A : \ \langle\psi|A\alpha\rangle = \langle \eta|\alpha\rangle
\ese
and hence $\psi\in\mathcal{D}_{A^*}$. Therefore, $\mathcal{D}_A\subseteq \mathcal{D}_{A^*}$ and $A^*\psi:=\eta = A\psi$.
\eq

\bd
A densely defined operator $A\cl\mathcal{D}_A\to\mathcal{H}$ is \emph{self-adjoint}\index{self-adjoint operator} if $A=A^*$. That is,
\ben[label=(\roman*)]
\item $\mathcal{D}_A = \mathcal{D}_{A^*}$
\item $\forall \, \alpha \in \mathcal{D}_A : \ A\alpha = A^*\alpha$.
\een
\ed

\br
Observe that any self-adjoint operator is also symmetric, but a symmetric operator need not be self-adjoint. 
\er

\bc
\label{cor:selfasdjext}
A self-adjoint operator is maximal with respect to self-adjoint extension.
\ec

\bq
Let $A,B$ be self-adjoint and suppose that $A\subseteq B$. Then
\bse
A \subseteq B = B^* \subseteq A^* = A
\ese
and hence, $B=A$.
\eq
In fact, self-adjoint operators are maximal even with respect to symmetric extension, for we would have $B\subseteq B^*$ instead of $B=B^*$ in the above equation. 

\subsection{Closability, closure, closedness}


\bd
\ben[label=(\roman*)]
\item A densely defined operator $A$ is called \emph{closable} if its adjoint $A^*$ is also densely defined.
\item The \emph{closure} of a closable operator is $\overline{A}:=A^{**}=(A^*)^*$.
\item An operator is called \emph{closed}\index{closed operator} if $A=\overline{A}$.
\een
\ed

\br
Note that we have used the overline notation in several contexts with different meanings. When applied to complex numbers, it denotes complex conjugation. When applied to subsets of a topological space, it denotes their topological closure. Finally, when applied to (closable) operators, it denotes their closure as defined above.
\er

\bp
A symmetric operator is necessarily closable.
\ep

\bq
Let $A$ be symmetric. Then, $A\subseteq A^*$ and hence, $\mathcal{D}_A\subseteq \mathcal{D}_{A^*}$. Since a symmetric operators are densely defined, we have
\bse
\mathcal{H}=\overline{\mathcal{D}_A}\subseteq \overline{\mathcal{D}_{A^*}} \subseteq \mathcal{H}.
\ese
Hence, $\overline{\mathcal{D}_{A^*}} = \mathcal{H}$.
\eq

Note carefully that the adjoint of a symmetric operator need \emph{not} be symmetric. In particular, we cannot conclude that $A^*\subseteq A^{**}$. In fact, the reversed inclusion holds.

\bp
If $A$ is symmetric, then $A^{**}\subseteq A^*$.
\ep

\bq
Since $A$ is symmetric, we have $A\subseteq A^*$. Hence, $A^{**}\subseteq A^*$ by \Cref{prp:adjointreverse}.
\eq

\bl
\label{lem:closableext}
For any closable operator $A$, we have $A\subseteq A^{**}$.
\el

\bq
Recall that $\mathcal{D}_{A^*}:=\{\psi\in \mathcal{H}\mid \forall\, \alpha\in \mathcal{D}_{A} : \langle \psi | A\alpha \rangle=\langle \eta | \alpha \rangle\}$. Then
\bse
\forall \, \psi \in\mathcal{D}_{A^*} : \forall \, \alpha \in \mathcal{D}_{A} : \  \langle \psi | A\alpha \rangle =  \langle A^* \psi | \alpha \rangle.
\ese
Since $\psi$ and $\alpha$ above are ``dummy variables'', and the order of quantifiers of the same type is immaterial, we have
\bse
\forall \, \psi \in\mathcal{D}_{A} : \forall \, \alpha \in \mathcal{D}_{A^*} : \  \langle \alpha | A\psi \rangle =  \langle A^* \alpha | \psi \rangle.
\ese
By letting $\eta:=A^*\psi$ in the definition of $\mathcal{D}_{A^{**}}$, we see that $\mathcal{D}_{A}\subseteq\mathcal{D}_{A^{**}}$. Moreover, by definition, $A^{**}\psi:=\eta := A\psi$, and thus $A\subseteq A^{**}$.
\eq

\bc
If $A$ is symmetric, then $A\subseteq \overline{A}\subseteq A^*$.
\ec


\bc
If $A$ is symmetric, then $\overline{A}$ is symmetric.
\ec

\bt
Let $A$ be a densely defined operator.
\ben[label=(\roman*)]
\item The operator $A^*$ is closed
\item If $A$ is invertible, we have $(A^{-1})^*=(A^*)^{-1}$
\item If $A$ is invertible and closable and $\overline{A}$ is injective, then $\overline{A^{-1}}=\overline{A}^{\,-1}$.
\een
\et

\subsection{Essentially self-adjoint operators}

Usually, checking that an operator is symmetric is easy. By contrast, checking that an operator is self-adjoint (directly from the definition) requires the construction of the adjoint, which is not always easy. However, since every self-adjoint operator is symmetric, we can first check the symmetry property, and then determine criteria for when a symmetric operator is self-adjoint, or for when a self-adjoint extension exists.

Two complications with the extension approach are that, given a symmetric operator, there could be \emph{no} self-adjoint extension at all, or there may be \emph{several} different self-adjoint extensions. There is, however, a class of operators for which the situation is much nicer.

\bd
A symmetric operator $A$ is called \emph{essentially self-adjoint}\index{essentially self-adjoint operator} if $\overline{A}$ is self-adjoint.
\ed

This is weaker than the self-adjointness condition.

\bp
If $A$ is self-adjoint, then it is essentially self-adjoint. 
\ep

\bq
If $A = A^*$, then $A\subseteq A^*$ and $A^* \subseteq A$. Hence, $A^{**}\subseteq A^*$ and $A^*\subseteq A^{**}$, so $A^* = A^{**}$. Similarly, we have $A^{**} = A^{***}$, which is just $\overline{A}=\overline{A}^{\,*}$.
\eq

\bt
If $A$ is essentially self-adjoint, then there exists a unique self-adjoint extension of $A$, namely $\overline{A}$.
\et

\bq
\ben[label=(\roman*)]
\item Since $A$ is symmetric, it is closable and hence $\overline{A}$ exists.
\item By \Cref{lem:closableext}, we have $A\subseteq \overline{A}$, so $\overline{A}$ is an extension of $A$. 
\item Suppose that $B$ is another self-adjoint extension of $A$. Then, $A\subseteq B = B^*$, hence $B^{**}\subseteq A^*$, and thus $A^{**}\subseteq B^{***}=B$. This means that $\overline{A}\subseteq B$, i.e.\ $B$ is a self-adjoint extension of the self-adjoint operator $\overline{A}$. Hence, $B=\overline{A}$ by \Cref{cor:selfasdjext}.\qedhere
\een
\eq

\br
One may get the feeling at this point that checking for essential self-adjointness of an operator $A$, i.e.\ checking that $A^{**}=A^{***}$, is hardly easier than checking whether $A$ is self-adjoint, that is, whether $A=A^*$. However, this is not so. While we will show below that there is a sufficient criterion for self-adjointness which does not require to calculate the adjoint, we will see that there is, in fact, a necessary and sufficient criterion to check for essential self-adjointness of an operator without calculating a single adjoint.
\er

\br
If a symmetric operator $A$ fails to even be essentially self-adjoint, then there is either no self-adjoint extension of $A$ or there are several. 
\er

\bd
Let $A$ be a densely defined operator. The \emph{defect indices}\index{defect indices} of $A$ are
\bse
d_+:=\dim (\ker(A^*-\mathrm{i})) , \qquad \quad d_- :=\dim (\ker (A^*+\mathrm{i})),
\ese
where by $A^*\pm\mathrm{i}$ we mean, of course, $A^*\pm\mathrm{i}\cdot\id_{\mathcal{D}_A}$.
\ed

\bt
A symmetric operator has a self-adjoint extension if its defect indices coincide. Otherwise, there exist no self-adjoint extension.
\et

\br
We will later see that if $d_+=d_-=0$, then $A$ is essentially self-adjoint.
\er

\subsection{Criteria for self-adjointness and essential self-adjointness}


\bt
A symmetric operator $A$ is self-adjoint if (but not only if)
\bse
\exists \, z\in\C :\ \ran(A+z)=\mathcal{H}=\ran(A+\overline{z}).
\ese
\et

\bq
Since $A$ is symmetric, by \Cref{prp:symext}, we have $A\subseteq A^*$. Hence, it remains to be shown that $A^*\subseteq A$. To that end, let $\psi\in \mathcal{D}_{A^*}$ and let $z\in \C$. Clearly, 
\bse
A^*\psi+\overline{z}\psi\in\mathcal{H}.
\ese
Now suppose that $z$ satisfies the hypothesis of the theorem. Then, as $\ran(A+\overline{z})=\mathcal{H}$,
\bse
\exists \, \alpha\in\mathcal{D}_{A} : \ A^*\psi+\overline{z}\psi = (A+\overline{z})\alpha.
\ese
By using the symmetry of $A$, we have that, for any $\beta\in\mathcal{D}_A$,
\bi{rCl}
\langle \psi| (A+z)\beta \rangle & = & \langle (A+z)^*\psi| \beta \rangle \\
 & = & \langle A^*\psi+\overline{z}\psi| \beta \rangle \\
 & = & \langle (A+\overline{z})\alpha | \beta \rangle  \\
 & = & \langle A\alpha | \beta \rangle +z\langle \alpha | \beta \rangle \\
 & = & \langle \alpha | A\beta \rangle +\langle \alpha | z\beta \rangle \\
 & = & \langle \alpha | (A+z)\beta \rangle.
\ei
Since $\beta\in\mathcal{D}_A$ was arbitrary and $\ran(A+z)=\mathcal{H}$, we have
\bse
\forall \, \varphi \in \mathcal{H} : \quad \langle \psi | \varphi\rangle= \langle \alpha | \varphi \rangle.
\ese
Hence, by positive-definiteness of the inner product, we have $\psi\in\alpha$, thus $\psi\in\mathcal{D}_A$ and therefore, $A^*\subseteq A$.
\eq


\bt
A symmetric operator $A$ is essentially self-adjoint if, and only if, 
\bse
\exists \, z\in \C\setminus \R : \ \overline{\ran(A+z)}=\mathcal{H}=\overline{\ran(A+\overline{z})}. 
\ese
\et

The following criterion for essential self-adjointness, which does require the calculation of $A^*$, is equivalent to the previous result and, in some situations, it can be easier to check.

\bt
A symmetric operator $A$ is essentially self-adjoint if, and only if,
\bse
\exists \, z\in \C\setminus \R : \ \ker(A^*+z)=\{0\}=\ker(A^*+\overline{z}). 
\ese
\et

\bq
We show that this is equivalent to the previous condition. Recall that if $\mathcal{M}$ is a linear subspace of $\mathcal{H}$, then $\mathcal{M}^{\perp}$ is closed and hence, $\mathcal{M}^{\perp\perp}=\overline{M}$. Thus, by \Cref{prp:kerran}, we have
\bse
\overline{\ran(A+z)}  =  \ran(A+z)^{\perp\perp}
 =  \ker(A^*+\overline{z})^{\perp}
\ese
and, similarly,
\bse
\overline{\ran(A+z)} = \ker(A^*+z).
\ese
Since $\mathcal{H}^{\perp}=\{0\}$, the above condition is equivalent to
\bse\overline{\ran(A+z)}=\mathcal{H}=\overline{\ran(A+\overline{z})}.\qedhere
\ese
\eq





















